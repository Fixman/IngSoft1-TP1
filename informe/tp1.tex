\documentclass[a4paper]{artcle}

\usepackage[paper=a4paper, left=1.5cm, right=1.5cm, bottom=1.5cm, top=3.5cm]{geometry}
\usepackage[spanish,activeacute]{babel}
\usepackage[latin1]{inputenc}
\usepackage{amsthm}
\usepackage{amsmath}
\usepackage{amsfonts}
\usepackage{amssymb}
\usepackage{alltt}
\usepackage{graphicx} %Para incluir el logo de la UBA
\usepackage{caratula} %Para armar el cuadro de integrantes
\usepackage{multirow} %Para poder poner varias lineas juntas sin divisiones en una tabla
\usepackage[lined,ruled,linesnumbered]{algorithm2e}
\usepackage{algpseudocode}
\usepackage{scrextend}
\usepackage{blindtext}


%Cosas para escribir codigo fuente
%Fuente: http://en.wikibooks.org/wiki/LaTeX/Source_Code_Listings
\usepackage{listings}
\usepackage{color}

\setcounter{secnumdepth}{5}

\definecolor{mygreen}{rgb}{0,0.6,0}
\definecolor{mygray}{rgb}{0.5,0.5,0.5}
\definecolor{myorange}{rgb}{1,0.4,0.2}
\definecolor{myblue}{rgb}{0,0,0.65}

%Configuracion para los listings
\lstset{ %
  backgroundcolor=\color{white},   % choose the background color; you must add \usepackage{color} or \usepackage{xcolor}
  basicstyle=\small,        % the size of the fonts that are used for the code
  breakatwhitespace=false,         % sets if automatic breaks should only happen at whitespace
  breaklines=true,                 % sets automatic line breaking
  captionpos=b,                    % sets the caption-position to bottom
  commentstyle=\color{mygreen},    % comment style
  deletekeywords={...},            % if you want to delete keywords from the given language
  escapeinside={\%*}{*)},          % if you want to add LaTeX within your code
  extendedchars=true,              % lets you use non-ASCII characters; for 8-bits encodings only, does not work with UTF-8
  frame=single,                    % adds a frame around the code
  keywordstyle=\color{myblue},       % keyword style
  language=Octave,                 % the language of the code
  morekeywords={*,...},            % if you want to add more keywords to the set
  numbers=left,                    % where to put the line-numbers; possible values are (none, left, right)
  numbersep=5pt,                   % how far the line-numbers are from the code
  numberstyle=\tiny\color{mygray}, % the style that is used for the line-numbers
  rulecolor=\color{black},         % if not set, the frame-color may be changed on line-breaks within not-black text (e.g. comments (green here))
  showspaces=false,                % show spaces everywhere adding particular underscores; it overrides 'showstringspaces'
  showstringspaces=false,          % underline spaces within strings only
  showtabs=false,                  % show tabs within strings adding particular underscores
  stepnumber=1,                    % the step between two line-numbers. If it's 1, each line will be numbered
  stringstyle=\color{myorange},     % string literal style
  tabsize=2,                       % sets default tabsize to 2 spaces
  title=\lstname                   % show the filename of files included with \lstinputlisting; also try caption instead of title
}

\renewcommand{\lstlistingname}{C\'{o}digo}

\lstset{language=C++,caption={Descriptive Caption Text},label=DescriptiveLabel}


%\topmargin = -1cm
%\textheight = 24cm 

\begin{document}

\integrante{Sclar, Melanie}{551/12}{melaniesclar@gmail.com}
\integrante{Zylber, Ariel}{530/12}{arielzylber@gmail.com}
\integrante{Hardy, Gonzalo}{449/09}{hardy.gonzalo@gmail.com}
\integrante{Fixman, Mart\'in}{391/11}{martinfixman@gmail.com}
\integrante{Aleman, Dami\'an Eliel}{377/10}{damianealeman@gmail.com}

\def\Materia{Ingenier\'ia del Software 1}
\def\Titulo{Trabajo Pr\'actico 1}
\def\Fecha{24 de septiembre de 2014}

%----- CARATULA -----%

\thispagestyle{empty}

\begin{center}
	\includegraphics[scale = 0.25]{logo_uba.jpg}
\end{center}

\vspace{5mm}

\begin{center}
	{\textbf{\large UNIVERSIDAD DE BUENOS AIRES}}\\[1.5em]
	{\textbf{\large Departamento de Computaci\'{o}n}}\\[1.5em]
    {\textbf{\large Facultad de Ciencias Exactas y Naturales}}\\
    \vspace{35mm}
    {\LARGE\textbf{\Materia}}\\[1em]    
    \vspace{15mm}
    {\Large \textbf{\Titulo}}\\[1em]
    \vspace{15mm}
    {\textbf{\Large \Fecha}}\\
    \vspace{15mm}
    \textbf{\tablaints}
\end{center}

\newtheorem{teo}{Teorema}[section]
\newtheorem{propo}{Proposici\'{o}n}[section]
\newtheorem{lema}{Lema}[section]
\newtheorem{coro}{Corolario}[section]
\newtheorem{defi}{Definici\'{o}n}[section]

\newpage
\thispagestyle{empty}
\tableofcontents

\parskip=5pt
\setlength{\parindent}{0pt}

\newpage
\setcounter{page}{1}
\pagenumbering{arabic}
\pagestyle{plain}

\newpage


\newcommand{\Asig}{\ensuremath{\leftarrow}}
\newcommand{\AndY}{\ensuremath{\wedge}}
\newcommand{\Or}{\ensuremath{\vee}}
\newcommand{\Not}{\ensuremath{\neg}}
\newcommand{\NotEq}{\ensuremath{\neq}}
\newcommand{\MayorIg}{\ensuremath{\geq}}
\newcommand{\tabu}{\hspace*{0.7cm}}
\newcommand{\ctabu}{\hspace*{0.8cm}}
\newcommand{\htabu}{\hspace*{0.35cm}}
\newcommand{\moduloNombre}[1]{\textbf{#1}}

\section{Aclaraciones}
\begin{itemize}
\item Cuando se utiliza el t\'ermino \textit{pedido}, puede ser tanto un pedido de viaje \'unico o un pedido de programaci\'on
de viaje regular.
\end{itemize}

\subsection{Sobre el tiempo de espera}
\begin{itemize}
\item Si se trata de un viaje regular, no se informa tiempo de espera porque en ese caso, lo \'unico que interesa es que llegue
a la hora pactada.
\item Si se trata de un viaje \'unico pedido por la aplicaci\'on, el tiempo de espera se informa cuando el taxi se comienza a movilizar
hacia destino (ya que podr\'a ver el trayecto del m\'ovil).
\item Si se trata de un viaje \'unico pedido por tel\'efono, el tiempo de espera se informa durante el llamado de pedido de taxi.
Si bien en ese momento no se sabe con certeza cu\'anto tardar\'a el taxi, el operador le da al cliente un tiempo aproximado.
La otra opci\'on que analizamos fue la de esperar hasta saber con certeza que un taxi se est\'a movilizando, pero en ese caso
se deber\'ian hacer 2 llamados por cada pasajero (dificultando m\'as el trabajo del operador).
\end{itemize}

\section{Diccionario}
\begin{itemize}
\item Viaje Simple: Denominamos viaje simple a un viaje solicitado por el usuario para realizarse una unica vez
\item Viaje Regular: Denominamos viaje regular a un viaje solicitado por el usuario, a realizarse periodicamente
\item Viaje: Es una generalizacion, aplica tanto para viajes regulares como para viajes simples.
\item Taxi adecuado: Denominamos taxi adecuado al taxi elegido para realizar el viaje, ya sea porque el usuario lo eligio a travez de la aplicacion, o porque el sistema determino automaticamente que es el taxi optimo para enviar.
\item Taxi movilizado: Un taxi adecuado pasa a ser movilizado cuando le taxista acepta realizar el viaje.
\end{itemize}

\section{Escenarios informales}

\subsection{Un usuario no registrado quiere pedir un taxi via web:}
\begin{itemize}
\item El usuario hace el pedido
\item El sistema verifica que el usuario no est\'a registrado y la aplicaci\'on le pide los datos
\item El usuario ingresa los datos y se registra
\item El usuario demanda el movil via la aplicaci\'on
\item La aplicaci\'on env\'ia el pedido al sistema
\item El sistema registra el pedido y muestra la informaci\'on sobre los taxis y taxistas disponibles.
\item El usuario elige el taxi
\item El sistema env\'ia la notificaci\'on del viaje nuevo
\item La terminal del m\'ovil notifica al taxista el viaje
\item El taxista confirma el viaje
\item La aplicaci\'on informa la confirmaci\'on del viaje y el tiempo aproximado de espera
\item La aplicaci\'on muestra la ubicaci\'on del taxi movilizado
\item Se realiza el viaje y el cliente califica al taxista
\end{itemize}

\subsection{Un usuario no registrado pide y luego cancela un viaje via tel\'efono:}
\begin{itemize}
\item El usuario hace el pedido
\item El operador verifica que el usuario no est\'a registrado en el sistema le pide los datos
\item El usuario le dice los datos y el operador los carga en el sistema
\item El usuario demanda un m\'ovil y el operador carga el pedido
\item El sistema registra el pedido y elige el mejor taxi disponible
\item El sistema env\'ia la notificaci\'on del viaje nuevo
\item La terminal del m\'ovil notifica al taxista el viaje
\item El taxista confirma el viaje
\item El operador le informa el tiempo aproximado de espera
\item El usuario solicita la cancelaci\'on del viaje (llama nuevamente por tel\'efono a la empresa)
\item El operador carga el pedido de cancelaci\'on del viaje
\item El sistema registra el pedido de cancelaci\'on del viaje y env\'ia la informaci\'on a la terminal del m\'ovil
\item La terminal del m\'ovil informa al taxista la cancelaci\'on del viaje y el taxista confirma la recepci\'on de la cancelaci\'on del viaje
\end{itemize}
(En realidad, en cualquier momento desde el pedido hasta que el taxista arriba al lugar donde se pidi\'o el taxi se puede cancelar el pedido, aqu\'i hacemos s\'olo un ejemplo de una situaci\'on hipot\'etica, las dem\'as son an\'alogas).

\subsection{Sobre los viajes regulares}
En este caso hicimos un O-refinamiento, dando dos posibles soluciones al problema de agendar (y enviar despu\'es, por supuesto)
viajes regulares. Las dos opciones son la de terciarizar el cumplimiento de los viajes regulares hacia una remiser\'ia (que se especializan en el cumplimiento eficiente de este tipo de viajes) o la de que TecnoTaxi tambi\'en se encargue del correcto env\'io
de estos viajes (generando posiblemente m\'as facturaci\'on, pero requiriendo una mayor sofisticaci\'on en el software).\\

S\'olo analizamos los casos de usuario registrado, ya que antes describimos c\'omo registrar un usuario por cualquiera de las dos v\'ias.

\subsubsection{Los viajes regulares son delegados a la remiser\'ia}

\paragraph{Un usuario registrado pide un viaje regular via tel\'efono:}
\begin{itemize}
\item El usuario llama por telefono
\item El opeardor verifica que el usuario est\'a registrado
\item El usuario le dice al operador la hora y la regularidad de los viajes que desea programar, asimismo indica punto de origen y lugar de destino
\item El operador carga los datos del viaje regular en sistema.
\item El sistema registra los pedidos 
\item Llegado el d\'ia anterior a la fecha del viaje programado, el sistema se comunica autom\'aticamente con el sistema de la remiser\'ia y encarga un veh\'iculo para realizar el viaje.
\item Se realiza el viaje
\end{itemize}


\paragraph{Un usuario registrado pide un viaje regular via la aplicaci\'on (y realiza alg\'un viaje):}
\begin{itemize}
\item El usuario se conecta a la aplicacion
\item El sistema verifica que el usuario est\'a registrado y no le pide sus datos personales nuevamente
\item El usuario programa una serie de pedidos utilizando la aplicaci\'on
\item La aplicaci\'on env\'ia los pedidos al sistema
\item El sistema registra los pedidos 
\item Llegado el d\'ia anterior a la fecha del viaje programado, el sistema se comunica autom\'aticamente con el sistema de la remiser\'ia y encarga un veh\'iculo para realizar el viaje.
\item Se realiza el viaje y el cliente califica al servicio
\end{itemize}

\subsubsection{Los viajes regulares son cumplidos por taxis de TecnoTaxi}

\paragraph{Un usuario registrado pide un viaje regular via tel\'efono:}
\begin{itemize}
\item El usuario llama por telefono
\item El opeardor verifica que el usuario est\'a registrado
\item El usuario le dice al operador la hora y la regularidad de los viajes que desea programar, asimismo indica punto de origen y lugar de destino
\item El operador carga los datos del viaje regular en sistema.
\item El sistema registra los pedidos 
\item Llegado el momento del env\'io de taxi, el sistema elige de manera autom\'atica el taxi que sera enviado.
\item El sistema env\'ia luego la notificaci\'on del viaje nuevo
\item La terminal del m\'ovil notifica al taxista el viaje
\item El taxista confirma el viaje
\item Se realiza el viaje
\end{itemize}


\paragraph{Un usuario registrado pide un viaje regular via la aplicaci\'on (y realiza alg\'un viaje):}
\begin{itemize}
\item El usuario se conecta a la aplicacion
\item El sistema verifica que el usuario est\'a registrado y no le pide sus datos personales nuevamente
\item El usuario programa una serie de pedidos utilizando la aplicaci\'on
\item La aplicaci\'on env\'ia los pedidos al sistema
\item El sistema registra los pedidos 
\item Llegado el momento del env\'io de taxi, el sistema elige de manera autom\'atica el taxi que sera enviado.
\item El sistema env\'ia luego la notificaci\'on del viaje nuevo
\item La terminal del m\'ovil notifica al taxista el viaje
\item El taxista confirma el viaje
\item La aplicaci\'on informa el taxi que fue asignado para el viaje, muestrando su ubicaci\'on
\item Se realiza el viaje y el cliente califica al taxista
\end{itemize}

\section{Aserciones}

\subsection{Aserciones descriptivas}

\begin{itemize}

\item El pasajero solicitara el taxi solo en caso de necesitarlo.
\item El pasajero espera como m\'aximo el plazo estimado
\item Responde solo el taxi disponible m\'as cercano
\item El taxista informa al llegar al domicilio
\item El gps informa constantemente la posici\'on correcta del taxi
\item El taxista tomara un camino m\'as corto
\item El pasajero llega a destino
\item El pasajero califica al taxista luego de haber realizado un viaje pedido via web/mobile
\item En caso de no necesitar mas el auto, el pasajero cancelar\'a el viaje.
\item El taxista confirma el viaje si est\'a disponible
\item El taxista informar\'a cuando se encuentre disponible para recibir un viaje
\item El taxista informar\'a cuando no se encuentre disponible para recibir un viaje
\item Los viajes programados que realiza la remiser\'ia ser\'an efectuados en tiempo y forma
\end{itemize}


\subsection{Aserciones prescriptivas}

\begin{itemize}
\item Para cada pedido de un cliente ir\'a solo un taxista a realizarlo
\item El tiempo de espera ser\'a m\'inimo, en funci\'on de los taxis disponibles
\item El auto elegido automaticamente ser\'a uno de los mejores calificados que est\'en en las cercan\'ias de el origen del viaje.
\item La cantidad de viajes que haya tenido un taxista con un pasajero ser\'a tenida en cuenta al momento de seleccionar los autos por el sistema de forma autom\'atica
\item Se le informar\'a a los clientes que pidieron un taxi via web la ubicaci\'on del taxi que realizar\'a el viaje.
 
\end{itemize}


\end{document}
